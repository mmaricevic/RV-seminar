\documentclass{beamer}
\usepackage[croatian]{babel}
\usepackage[utf8]{inputenc}
\usepackage{xcolor}
\usepackage{graphicx}
\usepackage{hyperref}
\usetheme{Ilmenau}
\definecolor{denim}{rgb}{0.08, 0.38, 0.74}
\usecolortheme[named=denim]{structure}
\author[Maričević, Prpić, Miculinić]{Mario Maričević | Rea Prpić | Valentina Miculinić}
\title{*Izrada prezentacija i postera u Latexu*}
\subtitle{Seminar iz kolegija Računalne vještine}
\institute{Tehnički fakultet,Rijeka}
\date{24.1.2017.}
\begin{document}

\begin{frame}
\titlepage
\end{frame}
\begin{frame}{Sadržaj}
\tableofcontents
\end{frame}

\section{Početci izrade prezentacije}
\subsection{Paketi}

\begin{frame}{Izbor paketa}
\begin{itemize}
	\item Na početku važno je odabrati jedan od ponuđenih paketa za izradu prezentacija.
	\item Neki od paketa koje možemo koristiti pri izradi su:
\end{itemize}
\begin{enumerate}
	\item \textbf{Beamer} 
	\item \textbf{Powerdot} 
	\item \textbf{Prosper... itd.}
\end{enumerate}
\begin{itemize}
	\item Svaki od njih posjeduje svoje prednosti i nedostatke.
\end{itemize}
\end{frame}

\subsection{Preambula}

\begin{frame}{Preambula prezentacije}
Osim deklaracije paketa, potrebno je definirati podatke vezane uz izradu:
\begin{itemize}
	\item Naslov prezentacije \pause
	\item Autora ili autore prezentacije \pause
	\item Ustanovu \pause
	\item Temu \pause
	\item Datum izrade...
\end{itemize} 
\end{frame}

\begin{frame}{Primjer preambule}
	\begin{figure}[h]
	\includegraphics[width=\textwidth,height=0.5\textheight]{picture1.png}
	\caption{Preambula u Beamer dokumentu}
	\end{figure}
 \end{frame}

 \subsection{Glavni dio dokumenta (body)}

 \begin{frame}{Osnovno o glavnom dijelu dokumenta}
 	\begin{itemize}
 		\item U glavnom dijelu dokumenta izrađujemo sadržaj prezentacije.
 		\item Dodajemo tekst,slike,videa,tablice ...
 		\item Uređujemo našu prezentaciju korištenjem raznih naredbi.
 		\item Dvije najvažnije stvari su izgled i sadržaj.
 		\item Sadržaj mora biti kratak i jasan,a izgled oku primamljiv.
    \end{itemize}
 \end{frame}

 \section{Svojstva paketa}

 \subsection{Beamer}

 \begin{frame}{Što je Beamer?}
 		\begin{itemize}
 			\item Beamer je paket za izradu prezentacije u Latexu koji nudi veliki izbor mogućnosti kao što su:
 		\end{itemize}
 		\begin{enumerate}
 			\item Kreiranje naslovnog slajda.
 			\item Kreiranje sadržajnog slajda.
 			\item Dodavanje dodatnih efekata u prezentaciju.
 			\item Isticanje pojedinih riječi ili rečenica.
 			\item Veliki izbor tema
 		\end{enumerate}
\end{frame}

\begin{frame}{Što je frame?}
		\begin{itemize}
			\item Prezentacija se sastoji od \textbf{frame-ova} koji nisu potpuno jednaki slajdovima.
		\end{itemize}
        \begin{alertblock}{Napomena!}	
        		\textbf{Jedan frame može sadržavati više slajdova.}
        \end{alertblock}
        \begin{block}{Primjer}
            Korištenje \textbf{pause} naredbe nakon određenog teksta.\newline
            Svako korištenje naredbe \textbf{pause} Latex prezentira kao novi slajd, međutim to je i dalje isti frame!
        \end{block}
\end{frame}

\begin{frame}{Korištenje naredbe Pause}
		\begin{figure}
		\includegraphics[width=0.3\textwidth,height=0.2\textheight]{picture2.png}
		\caption{Naredba pause}
		\end{figure}
		\begin{figure}
		\includegraphics[width=0.3\textwidth,height=0.3\textheight]{picture4.png}
		\includegraphics[width=0.3\textwidth,height=0.3\textheight]{picture5.png}
		\caption{Primjer dva slajda na jednom frame-u.}
		\end{figure}
\end{frame}

\begin{frame}{Kreiranje naslovnog slajda}
		\begin{itemize}
			\item Naslovni slajd kreiramo naredbom \textbf{titlepage}
		\end{itemize}
		\begin{alertblock}{Napomena!}	
        	Prije naredbe \textbf{titlepage} potrebno je definirati preambulu dokumenta.
        \end{alertblock}
        \begin{itemize}
        	\item Naslovni slajd sadržava temu,autore,podnaslov,datum izrade...
        \end{itemize}
\end{frame}

\begin{frame}{Kreiranje sadržajnog slajda}
		\begin{itemize}
			\item Sadržajni slajd kreira se naredbom \textbf{tableofcontents}.
			\item Sadrži sve sekcije i podsekcije dokumenta.
			\item Sekcije kreiramo naredbom \textbf{section},a podsekcije naredbom \textbf{subsection}.
		\end{itemize}
\end{frame}
	
\begin{frame}{Primjer sadržajnog slajda}
	\begin{figure}
		\includegraphics[width=0.7\textwidth,height=0.6\textheight]{picture9.png} 
		\caption{Popis svih sekcija i podsekcija dokumenta}
	\end{figure}
\end{frame}

\begin{frame}{Isticanje riječi ili rečenice}
	\begin{itemize}
		\item Postoje tri načina kako istaknuti ono što želimo:
	\end{itemize}
	\begin{enumerate}
		\item Naredbom \textbf{block} definiramo kutiju u boji prezentacije.
		\item Naredbom \textbf{alertblock} definiramo kutiju crvene boje koja predstavlja upozorenje.
		\item Naredbom \textbf{examples} definiramo kutiju zelene boje kojom dajemo primjere.
	\end{enumerate}
\end{frame}

\begin{frame}{Primjer isticanja teksta}
	 	\begin{figure}
		\includegraphics[width=\textwidth,height=0.3\textheight]{picture7.png} 
		\caption{Isticanje teksta u Beameru}
	\end{figure}
\end{frame}

\begin{frame}{Izbor teme}
	\begin{itemize}
		\item Beamer nudi širok izbor tema za izradu prezentacije.
		\item Boje tema možemo prilagoditi našim željama.
		\item Popis tema može se vidjeti \hyperlink{http://deic.uab.es/~iblanes/beamer_gallery/index_by_theme.html}{\underline{\textbf{ovdje}}}.
		\item Tema korištena u ovoj prezentaciji je Ilmenau.
	\end{itemize}
\end{frame}

\begin{frame}{Primjer dodavanja teme}
		 \begin{figure}
		\includegraphics[width=0.5\textwidth,height=0.1\textheight]{picture8.png} 
		\caption{Definiranje teme i boje u preambuli dokumenta}
	\end{figure}
\end{frame}

\subsection{Powerdot}

\begin{frame}{Powerdot}
  		\begin{itemize}
  			\item Powerdot je paket za izradu prezentacija u  Latexu sličan Beameru
  			\item Pruža nekoliko značajki koje olakšavaju izlagaču kreiranje slajdova profesionalnog izgleda u kratkom vremenskom razdoblju
  		\end{itemize}
 \end{frame}

 \begin{frame}{Osnovna uporaba}
 \textbf{mode=print}
 		\begin{itemize}
  	  		\item Ova se naredba može koristiti za ispis slajdova, brišu se slojevi i efekti prijelaza
 		\end{itemize}
 \textbf{paper=smartboard}
 		\begin{itemize}
     			\item Ovom naredbom formatiramo veličinu papira za korištenje na širokom zaslonu 
 		\end{itemize}
 \textbf{orient=landscape}
 		\begin{itemize}
     			\item Ovom naredbom postavljamo orijentaciju dokumenta
 		\end{itemize}
 \end{frame}

 \begin{frame}{Dodavanje bilješki}
 		\begin{itemize}
     			\item U powerdotu moguće je dodati bilješke na slajdove kao pomoć izlagaču
 		\begin{figure}
 			\includegraphics[width=0.7\textwidth,height=0.2\textheight]{picture10.png}
 			\caption{Prikaz korištenja bilješki u Powerdotu}
 		\end{figure}
 		\end{itemize}
 \end{frame}

 \begin{frame}{Stilovi i palete}
 		\begin{itemize}
    			\item Izgled Powerdot prezentacije može se promijeniti pomoću stilova i paleta
     			\item Stilovi mijenjaju ukupni izgled prezentacije dok palete određuju skup boja koji se koriste u stilu
 		\end{itemize}
 \end{frame}

 \begin{frame}{Prijelazni efekti}
 		\begin{itemize}
     			\item Prijelazni efekti mogu se dodati u prezentaciju kako bi bila vizualno privlačnija
 		\begin{figure}
 			\includegraphics[width=0.3\textwidth,height=0.3\textheight]{picture11.png}
 			\caption{Prikaz korištenja prijelaznih efekata u Powerdotu}
 		\end{figure}
 		\end{itemize}
 \end{frame}

 \begin{frame}{Overlays}
  		\begin{itemize}
     			\item Posebne naredbe mogu se koristiti za otkrivanje samo nekih elemenata slajdova umjesto cijelog sadržaja
 		\end{itemize}
 \textbf{pause}
 		\begin{itemize}
     			\item Prikazati će tekst iza naredbe nakon svih overlay-a na slajdu 
 		\end{itemize}
 \textbf{begin{itemize}[type=1]}
 		\begin{itemize}
     			\item Dodatni parametar omogućuje preklapanje overlay-a u itemize i enumerate okruženjima
 		\end{itemize}
 \end{frame}

 \begin{frame}{Citati}
 		\begin{figure}
 			\includegraphics[width=0.4\textwidth,height=0.4\textheight]{picture12.png}
 			\caption{Prikaz citata u Powerdotu}
 		\end{figure}   
 \end{frame}

 \begin{frame}{Usporedba Beamera i Powerdot-a}
 \textbf{Beamer}
 		\begin{itemize}
     			\item Puno tema i sofisticirani načini prilagodbe
     			\item Može se koristiti s LaTeX i također s pdfLaTeX
     			\item Podržava PNG, JPEG i PDF formate slike
 		\end{itemize}
 \textbf{Powerdot}
 		\begin{itemize}
     			\item Pruža predloške
     			\item Ne može se koristiti sa pdfLaTeX
 		\end{itemize}
\end{frame}

\section{Izrada postera}

\begin{frame}{Izrada postera}

    \begin{itemize}
        \item Možemo ih raditi uz pomoć dva paketa: tikzposter i beamerposter
        \item Oba koriste jednostavne komande, pružaju razne mogućnosti uređivanja i podržavaju velike formate papira
    \end{itemize}
    

    
\end{frame}
    
\begin{frame}{Tikzposter i Beamerposter}
    
    \textbf{Tikzposter}
    \begin{itemize}
        \item Kao i Beamerposter, koristi se za izradu postera u PDF formatu
        \item Klasa dokumenta koja obuhvaća projekte fancyposter i tikzposter
    \end{itemize}
    
 \textbf{Beamerposter}
\begin{itemize}
    \item Prednost: Beamerposter koristi istu sintaksu (iste komande) kao i Beamer prezentacija
    \item Nedostatak: Nema velik izbor tema pa je malo manje fleksibilan od tikzpostera
    \
\end{itemize}
    
\end{frame}


\subsection{Tikzposter}
\begin{frame}{Tikzposter: Preambula}

\begin{itemize}
    \item Nakon deklaracije klase dokumenta (tikzposter) treba postaviti i ostale parametre, npr.:
    \begin{itemize}
        \item veličina fonta (12pt, 14pt, 17pt, 20pt, 24pt...)
        \item veličina papira (a0, a1, a2...)
        \item orijentacija papira (landscape ili portrait)
        \item i dr.
    \end{itemize}
   \item  Kao i u prezentaciji, moramo definirati autora, datum, naslov i instituciju.

\end{itemize}

\end{frame}

\begin{frame}{}

\begin{itemize}
    \item Koristimo komandu \textbf{usetheme} kako bi zadali temu.
    \item Neke od tema su koje možemo koristiti su: \begin{itemize}
        \item Default		
        \item Rays	
        \item Basic	
        \item Simple
        \item Envelope 
        \item Wave 
        \item Board
        \item Autum 
        \item Desert		
    \end{itemize}
\end{itemize}
    
\end{frame}{}

\begin{frame}{Glavni dio dokumenta (body)}
\begin{itemize}
    \item Glavni dio dokumenta formiramo blokovima teksta (koristeći naredbu \textbf{block})
    \item Možemo mijenjati broj kolona (koristeći naredbu \textbf{column}) i njihovu širinu, što pruža puno fleksibilnosti
    \item Komandu \textbf{note} koristimo kad želimo zapisati neke bilješke \textit{preko} blokova teksta
    \item \textbf{\textit{NAPOMENA:}} standardna komanda za uvrštavanje slika \textit{ne funkcionira}, nego se mora koristiti naredba \textbf{tikzfigure}

\end{itemize}
    
\end{frame}

\subsection{Beamerposter}

\begin{frame}{Beamerposter: Preambula}

\begin{itemize}
    \item Preambula beamerpostera je gotovo ista kao i preambula beamer prezentacije
    \item Nakon deklaracije klase dokumenta i zadavanje osnovnih parametara, moramo se koristiti naredbom \textbf{usepackage} pomoću koje možemo zadati neke posebne parametre kao što su: \begin{itemize}
        \item orijentacija postera (landscape ili portrait)
        \item veličina postera (od a4 do a0, iako se dimenzije mogu i ručno namjestiti pomoću opcija \textit{width} i \textit{height})
        \item Nakon toga slijede standardne informacije za poster kao što su naslov, autor, datum i sl.
    \end{itemize}
\end{itemize}


\end{frame}

\begin{frame}{Glavni dio dokumenta (body)}

\begin{itemize}
    \item Da bi se kreirao poster sav sadržaj mora biti napisan unutar \textit{frame} okvira
    \item Glavnina sadržaja postera je u blokovima, kao i kod tikzpostera, a također se može koristiti naredba \textbf{column} za nove kolone
\end{itemize}
    
\end{frame}

\end{document}

